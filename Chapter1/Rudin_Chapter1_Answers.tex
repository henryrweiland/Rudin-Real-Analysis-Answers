\documentclass[12pt]{article}
\usepackage{graphicx}
\usepackage{bbm}
\usepackage{booktabs}
\usepackage{mathrsfs}
\usepackage[margin=1in]{geometry}
\usepackage{amsmath,amssymb,amsfonts,wrapfig,amsthm}
\title{Rudin, Principles of Mathematical Analysis,\\Chapter 1 Answer Key}
\date{}

\begin{document}
\maketitle

\begin{enumerate}
    \item If $r$ is rational ($r \neq 0$) and $x$ is irrational, prove that $r + x$ and $rx$ are irrational.
    
    $\textbf{Answer:}$ Assume otherwise that $r + x$ is rational with sum $\alpha$. We write $r + x = \alpha \implies x = \alpha - r$. This implies that $x$
    can be expressed as the difference of two rational rational numbers, $\alpha - r$, which is rational. Thus, $r + x$ must be irrational. $\rightarrow \leftarrow$

    We follow a similar argument for the second part. Assume otherwise that $rx$ is rational with product $\frac{\beta}{\gamma}$. We write $rx = \frac{\beta}{\gamma} \implies x = \frac{\frac{\beta}{\gamma}}{r}$. This
    implies that $x$ can be expressed as a (compound) fraction, which is violates the irrational property. Thus, $rx$ must be irrational. $\rightarrow \leftarrow$

    \item Prove that there is no rational number whose square is 12.
    
    $\textbf{Answer:}$ Assume otherwise that there exists a rational $p = \frac{m}{n}$ for $m,n$ not both even, such that:

    \[p^2 = 12\]

    We can re-arrange as follows:
    
    \[\left(\frac{m}{n}\right)^2 = 12\]

    \[m^2 = 12n^2\]

    Since $12$ is even, and the product of two terms involving an even number is always even, the RHS must be even.
    So, by the equality, $m^2$ is also even, which implies that $m$ is even and $n$ is odd.
    We can write $m = 2k$ for some integer $k$ and substitute:

    \[4k^2 = 12n^2\]

    \[k^2 = 3n^2\]

    We know that $n$ is odd, hence $3n^2$ is odd, which implies $k^2$ and $k$ are also odd. We can now express $k = 2j + 1$ and $n = 2l + 1$ for some integers $j, l$ and substitute:

    \[\left(2j + 1\right)^2 = 3\left(2l + 1\right)^2\]

    \[4j^2 + 4j + 1 = 3\left(4l^2 + 4l + 1\right)\]

    \[4j^2 + 4j + 1 = 12l^2 + 12l + 3\]

    \[4j^2 + 4j -  12l^2 - 12l = 2\]

    \[4\left(j^2 + j - l^2 - l\right) = 2\]

    This equality cannot hold because $4$ is not a multiple of $2$. Thus, $p$ must be irrational. $\rightarrow \leftarrow$

    
    
    \item Prove Proposition 1.15.
    
     $\textbf{Answer:}$ We want to prove the following statements that follow from the axioms of multiplication:

     \begin{itemize}
        \item If $x \neq 0$ and $xy = xz$ then $y = z$.
        
        \[xy = xz\]

        \[x^{-1}\left(xy\right) = x^{-1}\left(xz\right) \hspace{0.5cm} \text{Existence of an inverse element}\]

        \[\left(x^{-1}x\right)y = \left(x^{-1}x\right)z \hspace{0.5cm} \text{Associativity}\]

        \[(1)y = (1)z \hspace{0.5cm} \text{Definition of inverse}\]

        \[y = z \hspace{0.5cm} \text{Identity element}\]

        \item If $x \neq 0$ and $xy = x$ then $y = 1$.
        
        \[xy = x\]

        \[x^{-1}\left(xy\right) = x^{-1}\left(x\right) \hspace{0.5cm} \text{Existence of an inverse element}\]

        \[\left(x^{-1}x\right)y = \left(x^{-1} x\right) \hspace{0.5cm} \text{Associativity}\]

        \[(1)y = 1 \hspace{0.5cm} \text{Definition of inverse}\]

        \[y = 1 \hspace{0.5cm} \text{Identity element}\]

        \item If $x \neq 0$ and $xy = 1$ then $y = x^{-1}$.
        
        \[xy = 1\]

        \[x^{-1}\left(xy\right) = x^{-1}\left(1\right) \hspace{0.5cm} \text{Existence of an inverse element}\]

        \[\left(x^{-1} x\right) y = x^{-1} \hspace{0.5cm} \text{Associativity, identity element}\]

        \[(1)y = x^{-1} \hspace{0.5cm} \text{Definition of inverse}\]

        \[y = x^{-1} \hspace{0.5cm} \text{Identity element}\]

        \item If $x \neq 0$ then $\frac{1}{x^{-1}} = x$.
        
        \[\frac{1}{x^{-1}} = x\]

        \[x^{-1}\left(\frac{1}{x^{-1}}\right) = \left(x^{-1}x\right) \hspace{0.5cm} \text{Multiply by } x^{-1}\]

        \[1 = 1 \hspace{0.5cm} \text{Definition of inverse}\]

     \end{itemize}
    
    \item Let $E$ be a nonempty subset of an ordered set; suppose $x$ is a lower bound of $E$ and $\beta$ is an upper bound of $E$. Prove that $x \leq \beta$.
    
    $\textbf{Answer:}$ By Definition 1.7, since $\beta$ is an upper bound, then this must imply that $\alpha \leq \beta$ for all $\alpha \in E$. Simiarily, since $x$ is
    a lower bound, then this also must imply that $x \leq \alpha$ for all $\alpha \in E$. Hence, $x \leq \alpha \leq \beta$ and $x \leq \beta$ as desired.
    
    \item Let $A$ be a nonempty set of real numbers which is bounded below. Let $-A$ be the set of all numbers $-x$, where $x \in A$. Prove that
    \[
    \inf A = -\sup(-A).
    \]

    $\textbf{Answer:}$  Denote $\alpha = \inf A$. By definition 1.8, $\alpha$ is a lower bound of $A$ and no $x > \alpha$ is a lower bound of $A$.
    So, $\alpha \leq x, \forall x \in A$ implies $-\alpha \geq -x$ for $-x \in -A$, 
    which means $\alpha$ is an upper-bound of of $-A$. In order to show $-\alpha$ is the $\textit{least}$ upper bound of $-A$,
    take some $\beta < -\alpha$, which implies $-\beta > \alpha$. Since $\alpha = \inf A$, there exists some $x \in A$
    such that $\alpha < x < -\beta$ and $-x > \beta$, which implies $\beta$ is not an upper bound of $-A$. Hence, $-\alpha = \sup(-A)$ and $\alpha = -\sup(-A)$.
    
        \item Fix $b > 1$.
    \begin{enumerate}
        \item If $m, n, p, q$ are integers, $n > 0$, $q > 0$, and $r = m/n = p/q$, prove that
        \[
            (b^m)^{1/n} = (b^p)^{1/q}.
        \]
        Hence it makes sense to define $b^r = (b^m)^{1/n}$.

        \item Prove that $b^{r+s} = b^r b^s$ if $r$ and $s$ are rational.

        \item If $x$ is real, define $B(x)$ to be the set of all numbers $b^r$, where $r$ is rational and $r \leq x$. Prove that
        \[
            b^x = \sup B(x)
        \]
        when $r$ is rational. Hence it makes sense to define $b^x = \sup B(x)$ for every real $x$.

        \item Prove that $b^{x+y} = b^x b^y$ for all real $x$ and $y$.
    \end{enumerate}

    \item Fix $b > 1$, $y > 0$, and prove that there is a unique real $x$ such that $b^x = y$, by completing the following outline. (This is called the \emph{logarithm of $y$ to the base $b$}.)
    \begin{enumerate}
        \item For any positive integer $n$, $b^n - 1 \geq n(b - 1)$.

        \item Hence $b - 1 \geq n(b^{1/n} - 1)$.

        \item If $t > 1$ and $n > (b - 1)/(t - 1)$, then $b^{1/n} < t$.

        \item If $w$ is such that $b^{-x} < y$, then $b^{1/n} x < y$ for sufficiently large $n$; to see this, apply part (c) with $t = y \cdot b^x$.

        \item If $b^r > y$, then $b^{r - (1/n)} > y$ for sufficiently large $n$.

        \item Let $A$ be the set of all $w$ such that $b^w < y$, and show that $x = \sup A$ satisfies $b^x = y$.

        \item Prove that this $x$ is unique.
    \end{enumerate}

    \item Prove that no order can be defined in the complex field that turns it into an ordered field. \emph{Hint:} $-1$ is a square.

    \item Suppose $z = a + bi$, $w = c + di$. Define $z < w$ if $a < c$, and also if $a = c$ but $b < d$. Prove that this turns the set of all complex numbers into an ordered set. (This type of order relation is called a \emph{dictionary order}, or \emph{lexicographic order}, for obvious reasons.) Does this ordered set have the least-upper-bound property?

    \item Suppose $z = a + bi$, $w = u + iv$, and
    \[
        a = \left( \frac{|w| + u}{2} \right)^{1/2}, \quad b = \left( \frac{|w| - u}{2} \right)^{1/2}.
    \]
    Prove that $z^2 = w$ if $v \geq 0$ and that $(\bar{z})^2 = w$ if $v \leq 0$. Conclude that every complex number (with one exception!) has two complex square roots.
    
\end{enumerate}


\end{document}
